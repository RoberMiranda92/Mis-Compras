\apendice{Especificación de diseño}

\section{Introducción}

\section{Diseño de datos}

\section{Diseño procedimental}

\section{Diseño arquitectónico}
	\subsection{Cliente Android}
	En el caso del cliente para el desarrollo de la aplicación se ha elegido un patrón denominado \textit{MVP o Model-View-Presenter}.
	Este patrón permite la separación de la aplicación en componentes o lo que es lo mismo separar las vistas de la lógica de la aplicación.La principal ventaja de este patrón es que permite tener la misma lógica en varias vistas totalmente distintas \cite{mvpantonio}.
	
	\begin{itemize}
		\item \textbf{El modelo: }El modelo son los datos que se visualizan en la vista. O las clases que implementan la lógica del negocio y acceden a un sistema de datos persistente, un claro ejemplo son los DAO
		\item \textbf{La vista: } Es la interfaz que muestra los datos del modelo al usuario, generalmente implementada por un Activity o un Fragment, contiene una referencia a un presenter que se encargara de comunicar los eventos de la vista al modelo.
		\item \textbf{El Presentador: } Es el intermediario entre la vista y el modelo, se encarga de recuperar los datos del modelo y se los devuelve a la vista,por lo que tiene referencias tanto de la vista como del modelo. 
	\end{itemize}
	Podemos ver un esquema en la imagen \ref{fig:mvp2}, el modelo y la vista quedan completamente separados. Un pequeño ejemplo de como se desarrolla el ciclo de una aplicación se puede ver en la imagen \ref{fig:mvp-workflow}.
	

\imagen{mvp2}{Esquema de conexión MVP}
\imagen{mvp-workflow}{Ejemplo aplicación Android con MVP}

