\capitulo{7}{Conclusiones y Lineas de trabajo futuras}
\subsection{Conclusiones}
Las conclusiones que se pueden extraer del desarollo del presente trabajo pueden resumirse en:
\begin{itemize}
	\item Se ha obtenido una aplicación funcional, que a partir de una imagen de un tique se obtienen los diferentes artículos adquiridos. Hasta donde se conoce, este es el primer sistema de este tipo. Dado que se trata  una primera versión, existe mucho margen de mejora para futuras versiones, como se describe en \ref{lineasFuturas}.
	\item Debido al uso de varios campos para desarrollar el trabajo, se han utilizado una gran cantidad de conocimientos adquiridos en la carrera, y sobre todo nuevos campos como son la visión artificial junto con OpenCv y el desarrollo de aplicaciones bajo el Sistema Operativo Android.
\item Un aspecto muy relevante que se ha sacado de este proyecto es el desarrollo del mismo, de como a pesar de llevar una planificación pueden surgir imprevistos que pueden ayudar o entorpecer el desarrollo del mismo.
\end{itemize}

\subsection{Lineas de trabajo futuras \label{lineasFuturas}}
Este proyecto representa una version inicial de una aplicación que debe evolucionar y ampliar sus funcionalidades.

Durante la fase de diseño se ha tenido muy en cuenta la realización de una aplicación que permita posteriores ampliaciones y mejoras. Se ha buscado crear una base sobre la que se pueda seguir como referencia a la hora de realizar trabajos parecidos o ampliar el mismo. Ha habido algunas ideas que no han podido ser incorporadas debido a la falta de tiempo, y que podrían ser implementadas en un futuro. A continuación se proponen algunas:

\begin{itemize}
	\item \textbf{Ampliación del tipo de tique:} A día de hoy, la aplicación solo se ha diseñado para funcionar con un solo tipo de tiques, sin embargo, lo ideal seria utilizarla con varios modelos para alcanzar un número de productos y establecimientos superior.
	\item \textbf{Categorización automática:} Durante la realización del proyecto, se estudio como poder categorizar los productos automáticamente, se ha estudiado la herramienta BabelNet, la cuál es una base de datos semántica, que relaciona algunas palabras  con su significado como por ejemplo pan:con alimento o algunas marcas con la compañía como San Miguel con cerveza. El principal inconveniente es que cada tipo de entidad tiene muchos atributos diferentes y usarla sería muy complicado.
	\item \textbf{Utilización de la Minería de datos:} Usando word2vec una palabra se transforma en un vector, este conserva la semántica original, por lo tanto vectores próximos entre si pertenecen a la misma categoría y están alejados de otros vectores que provienen de palabras que entrarían en otra categoría, Si vemos la imagen \ref{fig:vecinos}, las zonas rojas podrían corresponder a la categoría de carnes, mientras que la zona azul seria la categoría pescados.En función de donde el algoritmo sitúe la palabra de la que se desea conocer la categoría, la categoría seria una u otra.
	\imagen{vecinos}{Gráfico de vecinos mas cercanos}
	
	\item \textbf{Nuevas características:} Dado que los productos son almacenados en un sistema persistente, es posible realizar una base de datos global, nutrida por todos los clientes, lo que supondría un conocimiento preciso de los productos y precios de estos en cada momento, y por lo tanto realizar búsquedas de dónde se localizan los productos con menor coste en tiempo real. Sin embargo establecer una relación entre los productos que simbolizan lo mismo pero con nombres diferentes, desembocaría en un coste elevado.
\end{itemize}