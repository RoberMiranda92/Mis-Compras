\capitulo{1}{Introducción}

En la era de la información que nos encontramos, donde todos los servicios se encuentran a click de ratón, el papel impreso está destinado a desaparecer. Sin embargo el ser humano es precavido y tiende a guardar todos los datos, por ello en los últimos años se ha empezado a digitalizar  los viejos informes de papel organizados en archivadores y ser guardados en grandes bancos de datos, que permiten una organización mucho mas limpia y más eficaz a la hora de acceder y buscar estos datos.

 Cada día realizamos compras ya a sea a través de la pantalla de dispositivos electrónicos, o de una forma mas tradicional con el contacto cliente-vendedor. En todas estas transacciones se genera una factura o tique, siendo en estas ultimas en papel, los cuales guardamos en la cartera o en casa para poder realizar un control de nuestros gastos y si es posible ahorrar en futuras compras.
 
 Este tipo de documentos también pueden ser digitalizados, sin embargo con este proyecto se quiere ir un paso mas allá, permitiendo guardar en un sistema persistente de base de datos cada articulo o producto adquirido, de forma automática, lo que permitiría un ahorro considerable de tiempo del usuario a la hora de guardar y consultar sus gastos.
 
 Generalmente las aplicaciones de control de gastos que podemos encontrar en el mercado de aplicaciones, en el caso de Android, Google Play emplean un sistema manual de almacenamiento de los datos, lo que hace que el usuario generalmente deje de emplear este tipo de sistema, ya que por naturaleza, las personas somos perezosas.
 
 La automatización del guardado de estos datos no solo ahorraría tiempo al usuario, si no que todos estos datos almacenados seria de gran importancia para las grandes compañías que realizan estudios de mercado y tendencias de compra, por lo tanto tendría un gran valor económico.
 
 La realización del presente trabajo se encuentra  dentro del campo de la visión artificial,el cual, se centra en la obtención de los datos a partir de una imagen.
 
 
 
