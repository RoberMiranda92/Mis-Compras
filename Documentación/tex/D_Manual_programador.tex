\apendice{Documentación técnica de programación}

\section{Introducción}

\section{Estructura de directorios}

\section{Manual del programador}
En esta sección se pretende explicar paso a paso como instalar, compilar y ejecutar el proyecto, como referencia a futuros programadores que usen este proyecto como base para desarrollar nuevas funcionalidades o nuevos proyectos.
Esos manuales se han desarrollado en un sistema operativo Windows 7 de 64 bits pero también se puede usar la configuración de 32 bits.
\section{Instalación, compilación y ejecución del proyecto}

\subsection{Instalación del Entorno.}

	\subsubsection{Java JDK 7}
	Para la ejecución del proyecto, es necesario el uso de Java, por ello se debe realizar una descarga desde la web \cite{java}, eligiendo la versión del sistema operativo en la que ejecutemos el proyecto.
	\subsection{NetBeans}
	El servidor, se ha desarrollado bajo el IDE de NetBeans, para instarlo, se debe ir a la web \cite{netbeans}, y seleccionar la version Java EE(ver figura \ref{fig:netBeansJava}).
	Una vez descargado instalar, y asegurarse de que se selecciona la versión \textit{GlassFish Server OpenSource Edition}, que instalará de forma automática GlassFish en nuestro ordenador.
	
\imagen{netBeansJava}{Selección de la versión de NetBeans}
	\subsection{Tesseract}
	Para el uso de Tesseract en el servidor, es necesario su instalación, pero \textbf{previamente} se debe instalar el paquete de \textbf{Visual C++ para visual Studio 2013}, que se puede descargar de \cite{paqueteVisual}.Se debe elegir la versión en función de la configuración del sistema operativo siendo:
	\begin{itemize}
		\item \textbf{vcredist\_x64.exe} para la versión de 64 bits.
		\item \textbf{vcredist\_x86.exe} para la versión de 32 bits.
	\end{itemize}
Una vez instalado procedes a la descarga de Tesseract de \cite{tesseract} donde elegimos el archivo \textbf{tesseract-ocr-setup-3.02.02.exe}(ver figura \ref{fig:tesseractDownload})

\imagen{tesseractDownload}{Selección de descarga Tesseract}
	
	\subsection{Android Studio}
	Para el desarrollo del cliente, se ha utilizado el IDE de Android Studio, para su instalación, se debe ir a la web \cite{androidStudio} y elegir la versión de sistema operativo.
	
	\subsection{Git}
	En el presentre proyecto se ha empleado repositorio Git para el control de versiones.Para la instalacion se debe descargar el programa de la web \cite{git} y completar la instalación.

\subsection{Importación del proyecto.}

\section{Pruebas del sistema}
