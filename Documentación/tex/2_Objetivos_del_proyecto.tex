\capitulo{2}{Objetivos del proyecto}

Con este proyecto se pretende conseguir una aplicación para smartphones basados en el sistema operativo  Android, con la que se pueda llevar un seguimiento de los gastos realizados por el usuario a través de una foto al tique.

\subsection{Objetivos Específicos}
Los objetivos dados por los requisitos software para el desarrollo del presente trabajo son los siguientes: 
\begin{itemize}
	\item Desarrollo de una aplicación para smartphones que permita llevar un historial de las compras realizadas por el usuario a partir de los tiques de compra.
	\item El origen de la imagen imagen del tique será de la galería de imágenes o de la cámara del smartphone.
	\item La aplicación sera programada en la plataforma Android.
	\item El historial de gastos será presentado de una manera clara y concisa, permitiendo ver rápidamente los consumos realizados.
	\item Se permitirá al usuario corregir las compras mal convertidas a partir de su correspondiente tique.
	\item Cada producto de un tique debe estar asignado a una categoría las cuales vendrán por defecto en la aplicación. 
	\item El usuario debe poder realizar diferentes consultas en función de los tique o productos guardados, pudiendo ser estas:
		\begin{itemize}
			\item Histórico de tique por fechas.
			\item Histórico de tique cuyo importe se encuentre en un rango.
			\item Productos comprados durante un periodo.
			\item Productos en un rango de precio.
			\item Productos de una categoría.
		\end{itemize}
\end{itemize}
Se pretenderá que la aplicación sea re-utilizable. Para ello se debe realizar una clara separación de los componentes 
\subsection{Objetivos Personales}
Los siguientes objetivos son las competencias que el alumno desea adquirir con el desarrollo del presente trabajo.
\begin{itemize}
	\item Aprendizaje de la plataforma Android como lenguaje de desarrollo de aplicaciones móviles. 
	\item Uso de OpenCV como librería de tratamiento de imágenes.
	\item Funcionamiento de un OCR.
	\item Desarrollo de un modelo Cliente-Servidor.
\end{itemize}

