\capitulo{6}{Trabajos relacionados}

Se ha procedido a realizar una búsqueda de aplicaciones similares para la organización de tiques.

Las aplicaciones se han buscado en la tienda de aplicaciones Google Play de Android.

\subsection{Smart Receipts}
Smart Receipts es una aplicación que permite almacenar las imágenes de cada tique y calcular el importe total de estos.
El importe se saca de manera automática del tique, sin embargo no se tiene un desglose detallado de los artículos que se encuentran en la imagen.Como característica interesante permite exportar los datos almacenados a PDF o a un archivo CSV.

Link a la app : \url{https://play.google.com/store/apps/details?id=wb.receipts&hl=es} 

\subsection{Receipts by Wave}
Receipts by Wave permite guardar las fotos de tus  facturas, para poder ahorrarte el tener que guardar cientos de papeles.
Dispone tanto de aplicación web y móvil, sin embargo el importe total se debe añadir de forma manual.

Link a la app : \url{https://play.google.com/store/apps/details?id=com.waveaccounting.receipts&hl=es} 

\subsection{Num receipts}
Num receipt se encarga de llevar un control sobre tus gastos, el sistema funciona guardado una foto e introduciendo el importe total.
La gran ventaja de esta aplicación es su sistema de representación gráfico, que permite ver a lo largo del tiempo los gastos del usuario.

Link a la app : \url{https://play.google.com/store/apps/details?id=com.skyvin.numr.scanreceipt&hl=es} 

\subsection{Fintonic}
Fintonic es una aplicación para dispositivos Android e IOs que se encarga de llevar tus gastos personales. Su uso se basa en introducir tu cuenta bancaria y consultar los movimientos de esta.

Link a la app: \url{https://play.google.com/store/apps/details?id=com.fintonic&hl=es}

\tablaSmall{Aplicaciones relacionadas}{l c c c c}{herramientasportipodeuso}
{ \multicolumn{1}{l}{Nombre} & Escaneo Auto& Desglose Productos & Precio\\}{ 
Smart Receipts &No&No&Gratis &\\
Receipts by Wave &Si & No & Gratis&\\
Num receipts & Si & No & Gratis-Pago&\\
Fintonic & No & No & Gratis&\\
}

Como podemos ver en la tabla \ref{tabla:herramientasportipodeuso}, si que existen aplicaciones que permiten el escaneo de tiques o facturas, sin embargo ninguna realiza un desglose por productos. Estas solo permiten llevar un control de los gastos totales por factura, lo que limita al usuario saber que ha consumido a menos que consulte la foto que realizo al tique.

Una aplición curiosa es el caso de Menta\cite{menta} la cual es una aplicación de descuentos.
Su funcionamiento es muy simple, cada día hay una serie de productos en oferta, si has adquirido dicho producto, con hacerle una simple foto, el sistema detecta si el producto se encuentra en la imagen.




